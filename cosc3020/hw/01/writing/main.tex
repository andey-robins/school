\documentclass[12pt]{article}

\usepackage{amsmath}
\usepackage[margin=1in]{geometry} % PAGE DIMENSIONS
\usepackage{listings}
\renewcommand{\familydefault}{\sfdefault}

\begin{document}
\pagenumbering{gobble}
%\noindent {\large Math 2250-04\hfill Name: \rule{7cm}{0.15mm}\\ Linear Algebra}
\noindent{\large COSC 3020 \hfill Name: \underline{Jacob Tuttle}\\
            Algorithms}\begin{center}
{\Large Homework 1}\\ \today{}
\end{center}
\vspace{5pt}
\textbf{1. Asymptotic Complexity:} Let $S$ be some arbitrary set such that $S \in O(log_2(n))$. \\

$S(n) \in O(log_2(n)) \implies S(n) \leq c \cdot log_2(n)$ for some arbitrary $c$.\\

Now, let $c$ be some value such that both $c \cdot log_2(n)$ and $c \cdot log_{10}(n)$ are greater than or equal to $S(n)$. \\

$S(n) \leq c \cdot log_{10}(n)$ as a result of our definition of $c$, and $S(n) \leq c \cdot log_{10}(n) \implies S(n) \in O(log_{10}(n))$. \\

Now consider $T(n)$ such that $T(n) \in O(log_{10}(n))$. \\

$T(n) \in O(log_{10}(n)) \implies T(n) \leq c \cdot log{10}(n)$ \\

Using the same arbitrarily large $c$ as above, we can show that: \\
\begin{align*}
    T(n) &\leq c \cdot log_2(n) \\
    &\implies T(n) \in O(log_2(n))
\end{align*}

Since $T(n) = c \cdot log_{10}(n) = S(n)$
\begin{equation*}
    T(n) = S(n) \implies O(log_{10}(n)) = O(log_2(n))
\end{equation*} \\

\textbf{2. Runtime Analysis:} Before being able to perform a runtime analysis evaluation, first the following code must be analyzed for complexity.
\begin{equation}
\texttt{if (n <= 1) \\}
\texttt{~return;}
\end{equation}
\begin{equation}
\texttt{mystery(n/3)}
\end{equation}
\begin{equation}
\texttt{for (var i = 0; i < n*n; i++) \{ count = count + 1; \}}
\end{equation} \\

These equations have runtimes as presented in the table below.

\begin{center}
\begin{tabular}{c|c}
Equation & Time Complexity \\
\hline
1 & $1$ \\
2 & $T(\frac{n}{3})$ \\
3 & $n^2$
\end{tabular}
\end{center}

Using these analyzed times, we can then constuct a piecewise function to define the time complexity of the function \texttt{mystery} in terms of the input size $n$.

\[ T(n) = \begin{cases}
    1 & n \leq 1 \\
    2 T(\frac{n}{3}) + n^2 & n > 1
\end{cases} \]

From there, we are able to trace the recurrence relation step by step for some arbitrarily large $n$.

\begin{align*}
    T(n) &= T(1) + 2 T(\frac{n}{3} + n^2) \\
    &= 1 + 2 (T(1) + 2T(\frac{n}{9}) + \frac{n}{3}^2) + n^2 \\
    &= 3 + 4T(\frac{n}{9}) + \frac{2n^2}{9} + n^2 \\
    &= 3 + 4T(\frac{n}{9}) + \frac{11n^2}{9} \\
    &= 3 + 4 (T(1) + 2T(\frac{n}{27}) + \frac{n}{9}^2) + \frac{11n^2}{9} \\
    &= 7 + 8T(\frac{n}{27}) + \frac{103n^2}{81} \\
    &= \vdots \\
    &= (2^i - 1) + 2i \cdot T(\frac{n}{3^i}) + \lambda_{i-1}n^2
\end{align*} \\

This approximation for the runtime of the $i$th recurence of the loop makes use of an equation\footnote{This formula for part of the fractional was retrieved from the Online Encyclopedia of Integer Sequences, http://oeis.org/A016133. Paolo P. Lava, June 16, 2008} (4) for the fraction multiplied with $n^2$, indicated by $\lambda_{i-1}$ above. The $m$ present in the equation is the iteration number $i$ minus $1$ since the equation uses an input of $0$ to produce its first value.

\begin{equation}
\lambda_m = \frac{\frac{-2^{m + 1}}{7} + \frac{9^{m + 1}}{7}}{9^m} = \frac{-2^{m + 1} + 9^{m + 1}}{7 \cdot 9^m}
\end{equation} \\

Since this function, for some value of $n$ will split it up into thirds, compounding the factor each time, (i.e. the first split is $\frac{1}{3}$, second $\frac{1}{9}$, etc.) the total number of executions will be $i = log_3(n)$. Substituting this value for $i$ provides a recurence relation of:

\begin{align*}
    T(n) &= (2^{log_3(n)} - 1) + 2 \cdot log_3(n) \cdot T(\frac{n}{3^{log_3(n)}}) + \lambda_{log_3(n)-1}n^2 \\
    &= (2^{log_3(n)} - 1) + 2 \cdot log_3(n) \cdot T(\frac{n}{n}) + \lambda_{log_3(n)-1}n^2 \\
    &= (2^{log_3(n)} - 1) + 2 \cdot log_3(n) \cdot 1 + \lambda_{log_3(n)-1}n^2 \\
    &= (2^{log_3(n)} - 1) + 2 \cdot log_3(n) + \lambda_{log_3(n)-1}n^2
\end{align*} \\

Of the three families of functions represented ($2^{log_3(n)}$, $log_3(n)$, $n^2$), the function with the most rapid growth is $n^2$. Disregarding the constants represented by $\lambda$, we arrive at the conclusion that $T(n) \in O(n^2)$ \\

\textbf{3. Sorting --- Insertion Sort:}\\
Array at the beginning of Insertion Sort:
\begin{center}
\begin{tabular}{ccccccc}
1 &7 &1 &5 &3 &-1 &9 \\
1 & 7 & 1 & 5 & 3 & -1 & 9 \\
1 & 1 & 7 & 5 & 3 & -1 & 9 \\
1 & 1 & 5 & 7 & 3 & -1 & 9 \\
1 & 1 & 3 & 5 & 7 & -1 & 9 \\
-1 & 1 & 1 & 3 & 5 & 7 & 9 \\
-1 & 1 & 1 & 3 & 5 & 7 & 9
\end{tabular}{}
\end{center}
End of sorting \\

\textbf{4. Sorting --- Merge Sort:} \\

Code solutions to the recursive, in-place merge sort problem provided in \texttt{mergeSort.js}. \\

Test code provided in \texttt{mergeSortTest.js} \\

Here is my bullshit analysis \\

\textbf{5. Sorting --- Quicksort:} \\
test



\end{document}
