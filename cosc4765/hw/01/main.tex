\documentclass[12pt]{article}

\usepackage{amsmath, amssymb}
\usepackage[margin=1in]{geometry}
\usepackage{helvet}
\usepackage{tikz}
% \usepackage{figure}
\renewcommand{\familydefault}{\sfdefault}

\begin{document}
\def\assignment{Homework 01}

\pagenumbering{gobble}
\noindent{\large COSC 4765 \hfill Name: \underline{Jacob Tuttle} \\ Computer Security}
\begin{center}
    {\Large \assignment} \\ \textbf{\today}
\end{center}


\question{Part 1: Definitions}

\begin{tabularx}{\textwidth}{|l|X|X|}
    \hline
    \textbf{Term}   & \textbf{Definition}                                                 & \textbf{Example}                                                                  \\ \hline
    Dependable      & A reliable and secure system                                        & Expected users can perform expected tasks, but nothing else can happen by design  \\ \hline
    Reliability     & We can expect X behavior to be possible                             & Alice will be able to read this file                                              \\ \hline
    Security        & We can expect Y behavior to not be possible                         & A foreign government won't be able to read this file                              \\ \hline
    System          & A physical *thing* with more products and people added on top of it & A smartphone running an OS with an owner acting as the user                       \\ \hline
    Subject         & A person, physically                                                & Bob                                                                               \\ \hline
    Person          & A subject or legal person (i.e. company)                            & Charlie, Walmart                                                                  \\ \hline
    Secrecy         & Limiting the principals with access to information                  & Classifications such as Top Secret only being accessible to some principal agents \\ \hline
    Error           & A deviation from expected behavior                                  & Expect the response 200 but get 401                                               \\ \hline
    Failure         & A system that fails to act in the way it was designed to            & A bike not spinning the wheels while pedaling                                     \\ \hline
    Reliability     & Probability of failure                                              & The chance of your bike breaking                                                  \\ \hline
    Availability    & Probability a system is operational                                 & How often your bike functions as intended                                         \\ \hline
    Risk            & Probability of an accident                                          & The chance your bike will break                                                   \\ \hline
    Accident        & An unplanned event that results in some loss                        & The chain falling off during a bike ride                                          \\ \hline
    Hazard          & Conditions where a failure can lead to an accident                  & Riding your bike while the chain is ungreased                                     \\ \hline
    Critical System & A condition where a failure will result in an accident              & A CPU; if it breaks the computer won't function                                   \\ \hline
\end{tabularx}

\newpage
\question{Part 2: Example --- Gas Station}

A gas station/convenience store utilizes a number of security features to ensure the safety of personel and property. The interior of the store houses various kinds of cheap merchandise, employees, and items that require their access be restricted (i.e. tobacco and cash). Outside, customers are responsible for interacting with the gas systems themselves in a system that can be entirely contact-less.
\begin{enumerate}
    \item Security cameras are installed within and around the store to facilitate security. By placing cameras in open, visible locations, it creates a preventative measure by increasing the risk of unwanted behavior being traced down and punished. Furthermore, it allows for the reporting of crimes, the detection of illegal activity after the fact (such as shoplifting or other theft), and the recovery of stolen goods by providing more information to authorities.
    \item Cash registers ensure the security of money within the store in a number of ways. It acts as a form of access control, ensuring that only approved or authorized users can access the cash reserves of the store. It also makes it more difficult to steal cash, a fact that helps to disincentivize a break in which helps increase the safety and security of the personel and property at the store.
    \item Window storefronts help further deter unwanted actions. Since anyone could look into the store and see unwanted behavior happening, the chance of a security breach being reported immediately increases, ensuring that failing all other security measures, a swift response to problems is still possible.
    \item Gas shut offs ensure the safety of patrons as well as the safety of property. In the case of a gas spill caused by a pump failing to shut off properly, the supply can still be shut off to prevent further leakage.
\end{enumerate}

\end{document}