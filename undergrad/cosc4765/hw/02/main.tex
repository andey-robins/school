\documentclass[12pt]{article}

\usepackage{amsmath, amssymb}
\usepackage[margin=1in]{geometry}
\usepackage{helvet}
\usepackage{tikz}
% \usepackage{figure}
\renewcommand{\familydefault}{\sfdefault}

\begin{document}
\def\assignment{Homework 01}

\pagenumbering{gobble}
\noindent{\large COSC 4765 \hfill Name: \underline{Jacob Tuttle} \\ Computer Security}
\begin{center}
    {\Large \assignment} \\ \textbf{\today}
\end{center}


\question{1. Adversary}

A subdivision of "The Swamp," the hacktivism group Anonymous might be one of the most well known hacktivist groups in modern pop culture. Earlier this year the group created a fake page for Taiwan and attached it to the United Nation's website. Taiwan's independence from China, and the standoff between the two states, has meant the Taiwan hasn't been represented at the UN since 1971. Taiwanese independence is often characterized as a fight between democracy and the authoritarian regimes of China, giving this hacking a clear political motivation that seems to be consistent with other attacks attributed to Anonymous. Finding exact details on how the attack was accomplished have been difficult; however, speculation seems to center around the hacktivist group posing as a fake NGO in order to gain access to the United Nations Department of Economic and Social Affairs servers. This would have required a fair amount of social engineering and the other attacks against people (instead of machines) that we have touched briefly on this week and last. \\

\question{2. Equality}

Our systems certainly do not treat all humans as equals. Beyond the innequalities based on sex discussed in the book and in lecture, I have yet to see a system that addequately designs for the differently abled people of the world. Consider designing for the blind. Almost every computer system takes the ability of sight for granted. A basic login page (I reference the Snap! programming language login page as that was the last login page I had used at the time of writing) could implement some very simple accessibility features such as shortcuts to select text boxes with the keyboard in order to integrate functionality into their website to support it by default. Browsers could implement screen reader features into their software instead of relying on third party software that wouldn't be available on public computers. To my knowledge, current best practices for accessibility involve tagging page elements in ways that make them friendly to screen readers instead of seeking to directly include design for the blind, deaf, or otherwise differently abled. \\

\question{3. Designing for End-Users}

I think developers should certainly plan and accomodate for a system's end users. Even if we ignore the benefits of designing for end-users as a viable business practice, it clearly will have some security benefits as well. Take password and login pages for example. Keeping the users in mind and designing a page that encourages password best practices, while giving useful feedback as the user attempts to create an account not only helps draw the user into your system, but it also ensures security best practices for passwords on your webpage. When users change, I don't think a different user base is a reason not to continue to support the changing base. Code, and related systems, are continually updated to remove bugs and add functionality, and if the user base has been changing, the system should also change to support their continued ease of use for the new system. \\

\question{4. Influencing Others}

\begin{enumerate}
    \item Using push notifications and smart devices to get people to exercise exploits peoples desire for commitment and consistency. They set up the system to tell them to exercise, they have a commitment to working, and as they continue to follow the workout system, they want to continue their efforts. The Apple watch, telling you to get up and move about every few hours and tracking your steps as a percentage of a goal implements this practice.
    \item Scam sites will use the principle of distraction to try to get more money out of you. For instance, if you're attempting to log into your paypal account because you're distracted by the thought that your account is going to be locked soon unless you log in to the link you've been provided exploits this way to influence others. By getting you to think about something else, the ramifications about losing your account, you're distracted from any possible warning signs present during your actions.
    \item A novel way to use the herd principle to influence people's online spending habits would be to add a "friends" network to your Amazon or other online shopping system. Then, you can display purchases that a user's friends have previously made in order to convince them to buy the same thing "because several of their friends have already bought it!" If you see that your friends Alice and Bob have both bought this new self-help book, and you look up to and respect Alice and Bob, you might be convinced to buy soemthing you had no intention to buy otherwise.
\end{enumerate}

\end{document}