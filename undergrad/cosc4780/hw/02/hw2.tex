\documentclass[12pt]{article}

\usepackage{amsmath, amssymb}
\usepackage[margin=1in]{geometry}
\usepackage{helvet}
\usepackage{tikz}
% \usepackage{figure}
\renewcommand{\familydefault}{\sfdefault}

\begin{document}
\def\assignment{Homework 04}

\pagenumbering{gobble}
\noindent{\large COSC 4780 \hfill Name: \underline{Jacob Tuttle} \\ Principles of Programming Langauges}
\begin{center}
    {\Large \assignment} \\ \textbf{\today}
\end{center}

\textbf{Cartesian Product: }$ A \times B = \{(a, b) | a \in A $ and $ b \in B\} $ \\

\noindent\textbf{Relations: }A relation is a set of tuples all taken from the same Cartesian product. If the relation has two elements, it is known as a binary relation, and we can generalize a binary relation $R$ as $aRb$ instead of $a,b \in R$ even though both notations are correct. \\

\noindent\textbf{Functions: }A function is a special relation where for every element in $A$, it will asign to a unique element in $B$. A function can be either partial (where it assigns at most one $b$ value) or total (it yields a value for every argument in its domain). \\

\noindent\textbf{Properties of an Equivalence Relation: }In order to be an equivalence relation, a relation must be reflexive ($xRx $ for $ x \in A$), symmetric ($xRy -> yRx $ for $ x,y \in A$), and transitive ($xRy \wedge yRz -> xRz $ for $ x,y,z \in A$)

\end{document}
